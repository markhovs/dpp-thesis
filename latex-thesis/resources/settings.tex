%%%%%%%%%%%%%%%%%%%%%%%%%%%%%%%%%%%%%%%%%%%%%%%%%%%%%%%%%%%%%%%%%%%%%%%%%%%%%%%%
% SETTINGS
%%%%%%%%%%%%%%%%%%%%%%%%%%%%%%%%%%%%%%%%%%%%%%%%%%%%%%%%%%%%%%%%%%%%%%%%%%%%%%%%

\KOMAoptions{parskip=half}      % Set paragraph spacing
\setstretch{1.172413793103448}  % Set line spacing

% Adjust page layout:
\usepackage[a4paper,    % Paper size
    head=20mm,          % Header height
    top=28mm,           % Top margin
    bottom=10mm,        % Bottom margin
    inner=30mm,         % Inner margin (binding side)
    outer=20mm,         % Outer margin
    includefoot         % Include footer in page layout calculations
]{geometry}

%%%%%%%%%%%%%%%%%%%%%%%%%%%%%%%%% TUM colors %%%%%%%%%%%%%%%%%%%%%%%%%%%%%%%%%
% Primary Colors in RGB
\definecolor{TUMBlue}{RGB}{0,101,189}%      Pantone 300
\definecolor{TUMWhite}{RGB}{255,255,255}%
\definecolor{TUMBlack}{RGB}{0,0,0}%
% Secondary Colors in RGB
\definecolor{TUMBlue1}{RGB}{0,51,89}%       Pantone 540
\definecolor{TUMBlue2}{RGB}{0,82,147}%      Pantone 301
\definecolor{TUMgray}{rgb}{0.498,0.498,0.498}
\definecolor{TUMGray1}{RGB}{51,51,51}%
\definecolor{TUMGray2}{RGB}{127,127,127}%
\definecolor{TUMGray3}{RGB}{204,204,204}%
% Accent Colors in RGB
\definecolor{TUMBlue3}{RGB}{100,160,200}%   Pantone 542
\definecolor{TUMBlue4}{RGB}{152,198,234}%   Pantone 283
\definecolor{TUMIvory}{RGB}{218,215,203}%
\definecolor{TUMOrange}{RGB}{227,114,34}%
\definecolor{TUMGreen}{RGB}{162,173,0}%

% Custom Colors for the Project
\usepackage[dvipsnames]{xcolor}
\definecolor{myGrey}{HTML}{B3B4BD}      % Grey
\definecolor{myDarkBlue}{HTML}{050A44}  % Dark Blue
\definecolor{myLightBlue}{HTML}{0A21C0} % Light Blue

%%%%%%%%%%%%%%%%%%%%%%%%%%%%%% Header and Footer %%%%%%%%%%%%%%%%%%%%%%%%%%%%%
\clearpairofpagestyles  % Clear default header and footer styles

% Header and Footer content
\automark[section]{chapter}     % 'section' title in even-page headers, 'chapter' title in odd-page headers
\ohead{\headmark}               % Set header content
\ofoot[\pagemark]{\pagemark}    % Set footer content with pagenumber

\setlength{\footheight}{15mm}   % Set footer height
\setkomafont{pagehead}{\fontsize{9pt}{10pt}\bfseries\normalfont}    % Set header font
\setkomafont{pagefoot}{\fontsize{12pt}{13pt}\bfseries\normalfont}   % Set footer font
%\setheadsepline{0.7pt}
\KOMAoptions{headsepline=0.7pt:\textwidth}


% Foonotes
\renewcommand{\footnoterule}{\hrule width 5.08cm height .6pt \vspace*{3.9mm}}   % appearance of horizontal line that separates main text from footnotes: Width 5.08 cm, height: 0.6 pt, after the rule there is a vertical space of 3.9 mm
\deffootnote{3mm}{3mm}{% Custom layout for the footnote: 3 mm indentation for firt line of the footnote, 3 mm indentation for subsequent lines
    \makebox[3mm][l]{\textsuperscript{\thefootnotemark}}    % box of 3 mm for the footnote number, formated as superscript
}
\setkomafont{footnoterule}{\fontsize{9pt}{20pt}\selectfont} % Font of footnote rule


%%%%%%%%%%%%%%%%%%%%%%%%%%%%%%%%%% Headings %%%%%%%%%%%%%%%%%%%%%%%%%%%%%%%%%%
\KOMAoptions{%
open=right,             % Start chapters on right-hand pages
    numbers=noendperiod, % No period after section numbers
    headings=small      % Small font for headings
}

\makeatletter
\g@addto@macro{\@afterheading}{\vspace{-\parskip}} % Remove axtra space after headings
\renewcommand*{\chapterheadstartvskip}{\vspace{\@tempskipa}\vspace*{-7mm}} % Adjust space above chapter headings
\makeatother

\setkomafont{disposition}{\bfseries\sffamily}   % Font for section commands (chapter, section, ...)

% Chapter
\setkomafont{chapter}{\normalfont\fontsize{18pt}{21pt}\selectfont}
\RedeclareSectionCommand[%
  beforeskip=0pt,               % No space before chapter title  
  afterskip=10pt,               % 10 pt space after chapter title
  tocbeforeskip=0pt plus .2pt,  % Space before chapter entry in table of content (toc)
  tocindent=0pt,                % No indent for chapter entry in toc
  tocnumwidth=25pt,             % Width for chapter number in toc
  toclinefill=\hfill            % Fill line after chapter number in toc
]{chapter}
\renewcommand*{\chapterformat}{\makebox[11mm][l]{\thechapter \enskip}}  % Chapter number

% Section
\setkomafont{section}{\fontsize{15pt}{18pt}\selectfont}
\RedeclareSectionCommand[%
  beforeskip=0pt,
  afterskip=10pt,
  tocindent=0pt,
  tocnumwidth=25pt,
  toclinefill=\hfill
]{section}
\renewcommand*{\sectionformat}{\makebox[11mm][l]{\thesection \enskip}} % Feste Breite für Abschnittsnummer und ohne Punkt danach

% Subsection
\setkomafont{subsection}{\fontsize{12pt}{15pt}\selectfont}
\RedeclareSectionCommand[%
  beforeskip=0pt,
  afterskip=5pt,
  tocindent=25pt,
  tocnumwidth=35pt,
  toclinefill=\hfill
]{subsection}
\renewcommand*{\subsectionformat}{\makebox[11mm][l]{\thesubsection \enskip}} % Feste Breite für Unterabschnittsnummer und ohne Punkt danach

% Subsubsection
\setkomafont{subsubsection}{\bfseries\fontsize{12pt}{15pt}\selectfont}
\RedeclareSectionCommand[%
beforeskip=0pt,
afterskip=5pt
]{subsubsection}
\renewcommand*{\subsubsectionformat}{\makebox[14mm][l]{\thesubsubsection.\enskip}} % Feste Breite für Unterabschnittsnummer und ohne Punkt danach

\setcounter{secnumdepth}{3}  % Enable numbering for subsubsections

% Color chapter titles
\addtokomafont{chapter}{\color{TUMgray}\bfseries}

% Line under chapter titles
\preto\chapterheadendvskip{%
    {\setlength{\parskip}{0pt}%
    \setlength{\parfillskip}{0pt plus 1fil} % 
    \setlength{\hfuzz}{12pt}                    % hfuzz allows line to be a bit longer than specified without generating overfull hbox warning
    \vspace{-5mm}                               % Reduce space above the line (closer to chapter title)  
    \noindent\color{TUMgray}\rule[-.3\baselineskip]{\linewidth}{0.5pt}}\par % horizontal line (rule), offset vertically by -.3\baselineskip, width equal to linewidth and height 0.5 pt
}

%%%%%%%%%%%%%%%%%%%%%%%%%%%%%%%%%%%% Lists %%%%%%%%%%%%%%%%%%%%%%%%%%%%%%%%%%%%

\setlist{nolistsep} %Reduce space between text and lists

% Customize lists:
\setlist{%
    labelsep=0mm,
    itemindent=0pt,
    labelindent=0pt,
    align=left,
    parsep=0.0ex,
    nosep
}
\setlist[itemize]{%
    leftmargin=5mm,
    labelwidth=4.9mm
}
\setlist[itemize,1]{        % First level itemize
    before={\vspace{-0.25ex}},
    label={\raisebox{.35ex}{\smaller[2]\textbullet}},
    %after={\vspace{-\parsep}\vspace{-.5ex}} %Absatzgröße nach Listen
}
\setlist[itemize,2]{        % Second level itemize
    label={\raisebox{.35ex}{\rule{.58ex}{.58ex}}}
}
\setlist[enumerate]{        % First level enumerate
    leftmargin=10mm,
    labelwidth=9.9mm
}
\setlist[enumerate,2]{      % Second level enumerate
    label={\alph*.}
}

\setlist[description]{%
%    labelindent=!,
    leftmargin=1em,
    labelwidth=!,
    parsep=0mm,
    partopsep=0mm,
    labelsep=1em,
}


%%%%%%%%%%%%%%%%%%%%%%%%%%%%%%%% Content lists  %%%%%%%%%%%%%%%%%%%%%%%%%%%%%%%
\KOMAoptions{%
    listof=totoc,       % Title of content list included in table of contents
    listof=nochaptergap,% Remove vertical space in toc
    listof=numbered,    % list of figures and tables are numbered
    listof=entryprefix, % Adds "Figure" / "Table" in lof/lot
}

% Table of content (toc)
\AfterTOCHead[toc]{\protect\vspace{-.1ex}\doublespacing} % Spacing between toc header and toc entries
\setcounter{tocdepth}{2}    % include up to subsections

% List of figures (lof)
\AfterTOCHead[lof]{\protect\vspace{-.1ex}\doublespacing} % Spacing between lof header and lof
\setuptoc{lof}{noparskipfake} % Adjust spacing after the lof header to match other headers

% List of tables (lot)
\AfterTOCHead[lot]{\protect\vspace{-.1ex}\doublespacing} % Spacing between lot header and lot
\setuptoc{lot}{noparskipfake} % Adjust spacing after the lot header to match other headers

%%%%%%%%%%%%%%%%%%%%%%%% Tables, Figures and Equations %%%%%%%%%%%%%%%%%%%%%%%%
\renewcommand{\thetable}{\thechapter -\arabic{table}}       % Table number = "chapter-number"
\DeclareCaptionLabelFormat{fmllabel}{#1 #2:\hspace{0.7cm}}  % Format for table caption
\captionsetup[table]{labelformat=fmllabel, singlelinecheck = off, labelfont = it, textfont = it, font = footnotesize, labelsep=none}    % Table captions

\renewcommand{\arraystretch}{1.5} % Scaling factor for tables

% New column types
\newcolumntype{M}{X<{\vspace{4pt}}} % Table with additional vertical spacing to improve readability
\newcolumntype{L}[1]{>{\hsize=#1\hsize\linewidth=\hsize}>{\raggedright\arraybackslash}X}% Aligned left, flexible width
\newcolumntype{R}[1]{>{\hsize=#1\hsize\linewidth=\hsize}>{\raggedleft\arraybackslash}X} % Aligned right, flexible width
\newcolumntype{C}[1]{>{\hsize=#1\hsize\linewidth=\hsize}>{\centering\arraybackslash}X}  % Aligned center, flexible width
\newcolumntype{P}[1]{>{\centering\arraybackslash}p{#1}}                                 % Aligned center, fixed width - Code example: P{4cm}
% Example: \begin{tabularx}{\textwidth}{|L{1}|C{1}|P{2cm}|M|}

% Figures
\renewcommand{\thefigure}{\thechapter-\arabic{figure}} 
\captionsetup[figure]{labelformat=fmllabel, singlelinecheck = off, labelfont = it, textfont = it, font = footnotesize, labelsep=none}
\graphicspath{{Abbildungen/}}

%Equations
\DeclareMathSymbol{*}{\mathbin}{symbols}{"01} %Ersetzt * mit Punkt bei Formeln
\renewcommand{\theequation}{\thechapter-\arabic{equation}}

%%%%%%%%%%%%%%%%%%%%%%%% Glossaries and Acronym list %%%%%%%%%%%%%%%%%%%%%%%%
\glsaddkey
 {unit}     % Add a unit key to glossary entries
 {}
 {\glsentryunit}    % lowercase     in glossary entry
 {\Glsentryunit}    % capitalized   in glossary entry
 {\glsunit}         % lowercase     in document
 {\Glsunit}         % capitalized   in document
 {\GLSunit}         % all caps      in document


\makeglossaries     % Enable glossaries
\setacronymstyle{long-short}
\renewcommand{\glossarysection}[2][]{}

% \newlength{\@glsdotsep} hat einen Fehler beim Kompilieren geworfen
% \setlength{\@glsdotsep}{\@dotsep em}
\newcommand*{\glsdotfill}{\leavevmode \cleaders \hb@xt@ \@glsdotsep{\hss .\hss }\hfill \kern \z@}
\makeatother

% Custom glossary style
\newglossarystyle{WissenschaftlicheArbeiten}{%
  \setglossarystyle{index}%

  \renewcommand*{\glossaryheader}{\vspace{.75em}}% Space before gls header
  \renewcommand*{\glstreenamefmt}[1]{##1}% Format for gls tree names
  \renewcommand*{\glossentry}[2]{%
     \item\glsentryitem{##1}\glstreenamefmt{\glstarget{##1}{\glossentryname{##1}}}% Format for glossary entry
     \ifglshassymbol{##1}{\space(\glossentrysymbol{##1})}{}% Include symbols if present
     \space-\space\glossentrydesc{##1}\glsdotfill\glspostdescription\space ##2% Format entry description with fot fill
  }%
  \renewcommand*{\glsgroupheading}[1]{%
    \item\glstreenamefmt{\textbf{\fontsize{14}{17}\selectfont\enskip\glsgetgrouptitle{##1}}}\vspace{.3em}}%
}

% Prevent orphans and widows (Schusterjungen und Hurenkinder)
\clubpenalty=10000                  % Prevent orphaned lines at the beginning of a page
\widowpenalty=10000                 % Prevent widow lines at the end of a page
\widowpenalties=3 10000 10000 150   % Additional setting to prevent widows and orphans