%==========================================
% Abstract (English) and Kurzfassung (German)
%==========================================

\chapter*{Abstract}
\thispagestyle{empty}

Digital Product Passports have emerged as a powerful regulatory initiative that aims to strengthen product lifecycle management, enhance traceability, and promote Circular Economy goals across diverse sectors. While regulators are introducing these passports to address persistent challenges in product transparency and sustainability, industry stakeholders must adopt technological solutions that guarantee compliance and open new avenues for circular business models.

The thesis adopts a multi-tier methodology to provide a comparative analysis of available Digital Product Passport technologies with the aim of identifying pathways for large-scale adoption. First, an extensive literature review establishes the conceptual framework, regulatory background, and core enabling technologies of Digital Product Passports, identifying requirements for viable Digital Product Passport systems and current research gaps. Next, a quantitative analysis of current Digital Product Passport initiatives, drawing on the CIRPASS database, reveals patterns of technology adoption and emerging industry trends. Expert interviews then provide qualitative insights into real-world challenges, forming the basis for a utility evaluation framework for Digital Product Passport systems based on key performance indicators such as interoperability, scalability, and cost efficiency.

Building on these findings, the thesis presents a generic system design model for Digital Product Passport solutions. This model is used to develop a prototype that demonstrates a modular, standards-based approach to data handling and governance. The previously developed utility value assessment validates the prototype, shedding light on practical considerations for broader industry implementation. The research concludes that federated digital twin systems, supported by robust data governance and scalability structures, appear most promising for facilitating widespread adoption across diverse sectors. Recommendations for further research include large-scale industrial testing and expanded standardization initiatives to fully realize the potential of Digital Product Passports in advancing sustainability and transparency.

\selectlanguage{ngerman}
\chapter*{Kurzfassung}
\thispagestyle{empty}

Digitale Produktpässe haben sich als einflussreiche regulatorische Initiative etabliert, um das Management von Produktlebenszyklen zu stärken, die Rückverfolgbarkeit zu verbessern und Ziele der Kreislaufwirtschaft in verschiedenen Branchen zu fördern. Während Regulierungsbehörden diese Pässe einführen, um anhaltende Defizite in puncto Produkttransparenz und Nachhaltigkeit zu beheben, sind Industrieakteure gefordert, technologische Lösungen umzusetzen, die einerseits die Einhaltung rechtlicher Vorgaben gewährleisten und andererseits neue Möglichkeiten für zirkuläre Geschäftsmodelle eröffnen.

Diese Arbeit verfolgt eine mehrstufige Methodik, um eine vergleichende Analyse verfügbarer Technologien für Digitale Produktpässe durchzuführen und mögliche Wege für deren großflächige Einführung aufzuzeigen. Zunächst werden im Rahmen einer umfassenden Literaturrecherche das konzeptionelle Fundament, der regulatorische Hintergrund sowie die wichtigsten Basistechnologien erörtert, wobei Anforderungen an funktionsfähige Systeme für Digitale Produktpässe und bestehende Forschungslücken herausgearbeitet werden. Anschließend beleuchtet eine quantitative Untersuchung aktueller Initiativen, die auf der CIRPASS-Datenbank basiert, Muster in der Technologieanwendung und neu aufkommende Branchentrends. Ergänzend liefern Experteninterviews praxisnahe Einblicke in reale Herausforderungen und dienen als Grundlage für eine Nutzwertanalyse, die zentrale Leistungsindikatoren wie Interoperabilität, Skalierbarkeit und Kosteneffizienz berücksichtigt.

Aufbauend auf diesen Erkenntnissen stellt die Arbeit ein generisches Systemdesign-Modell für Digitale Produktpass-Lösungen vor. Dieses Modell dient als Ausgangspunkt für die Entwicklung eines Prototyps, der einen modularen, standardbasierten Ansatz für Datenverarbeitung und Data Governance demonstriert. Die zuvor durchgeführte Nutzwertanalyse validiert den Prototyp und beleuchtet praxisrelevante Gesichtspunkte für eine breitere Implementierung in der Industrie. Die Untersuchung kommt zu dem Ergebnis, dass föderierte Systeme mit Digitalen Zwillingen, untermauert durch robuste Data Governance und skalierbare Strukturen, das größte Potenzial für eine branchenübergreifende Verbreitung bieten. Abschließend spricht die Arbeit Empfehlungen für weiterführende Forschungsaktivitäten aus, darunter groß angelegte Praxistests sowie erweiterte Standardisierungsinitiativen, um das volle Potenzial Digitaler Produktpässe im Hinblick auf Nachhaltigkeit und Transparenz auszuschöpfen.

% Switch back to English for the rest of the thesis
\selectlanguage{english}
