\chapter{Critical Evaluation and Future Outlook}
\label{cha:chapter_6}

\section{Research Outcomes and Discussion}

With the simultaneous advancement of digital technologies and the escalating urgency of sustainability challenges, the manufacturing industry finds itself at a critical intersection. Here, the convergence of Industry 4.0 principles and the collective societal shift toward circular economic models has catalyzed the emergence of the \acrlong{dpp} concept. Driven by clear economic and environmental motivations and further accelerated by regulatory mandates, particularly under the \ac{eu}’s \ac{espr} framework, \ac{dpp}s have spurred considerable interest and rapid growth, both industrially and academically. This regulatory-driven momentum, aiming toward comprehensive and unified product data tracking, has unveiled significant technological and architectural gaps. Numerous enabling technologies present potential pathways for globally scalable \ac{dpp} solutions, yet the inherent complexity and sector-specific needs complicate the feasibility of a universally applicable solution. Consequently, there is a pressing need to methodically identify comprehensive system requirements, rigorously evaluate suitable technological candidates, and ultimately develop a structured yet flexible framework. Such a framework must empower system designers to craft cohesive, compliant, and use-case-specific \ac{dpp} solutions, systematically assess these solutions against industry standards and empirical trends, and iteratively refine the resulting architectures to achieve optimal alignment with both regulatory demands and practical industry requirements.

\textbf{RQ1} was formulated in response to the fragmented technological landscape surrounding \ac{dpp}s. Its primary objective was to identify existing technological paradigms and frameworks and integrate them into a coherent and unified generic architecture model. To achieve this, Zwicky’s \acrlong{gma} method was rigorously adopted, providing a structured approach to navigate the multidimensional complexity inherent in \ac{dpp} system design. The resulting morphological box synthesized both the latest academic insights and empirical industry trends, capturing an extensive array of technological and architectural alternatives validated through a thorough literature review, quantitative analyses of industry initiatives, and expert interviews. This methodological rigor ensured that the resulting morphological box was not only robust but also deeply reflective of current industry practices and state-of-the-art research.

The derived generic model was subsequently operationalized and validated through the implementation of a proof-of-concept pilot, explicitly demonstrating the feasibility and real-world applicability of the developed conceptual framework. By applying the generic architecture in a practical scenario, the research highlighted the effectiveness and adaptability of the morphological approach in guiding system designers toward coherent architectural decisions—from foundational elements such as data storage and governance structures to advanced integrations like digital twins and semantic modeling.

However, despite its rigorous development and empirical grounding, the generic model carries inherent limitations reflective of its high-level and broadly applicable nature. Firstly, the model, by design, provides generalized architectural pathways, intentionally omitting highly specialized industry niches or technological boundary conditions that are prevalent in certain specific contexts. Industry applications frequently encounter highly individualized requirements, specialized legacy systems, or niche technological constraints that fall outside the broad strokes of a generic model. Therefore, while the morphological framework offers extensive configurational flexibility, specific industry applications might still necessitate additional custom adaptations and deeper technical refinement beyond the scope of the current framework.

Secondly, the generic model and its corresponding decision flow predominantly assume system design scenarios initiated from scratch. In practice, many industry scenarios involve pre-existing infrastructures, legacy data systems, or partial technological commitments. Such preconditions would necessitate consideration of reverse pathways or hybrid configurational approaches not explicitly accounted for within the current morphological flow. Addressing these retrospective or hybrid scenarios would require extensions or adaptations of the generic model to systematically manage architectural integration or migration from existing solutions.

Finally, although the current framework extensively outlines available technological and architectural configurations, it deliberately abstains from prescribing specific templates for common industry scenarios. Going a step further by systematically identifying, illustrating, and documenting representative scenario-specific "templates" or exemplary configurations could significantly enhance practical applicability and industry adoption. Such exemplary templates would provide a pragmatic starting point, reducing initial design complexity and facilitating rapid prototyping or adaptation across common industrial contexts. However, the development and rigorous validation of these scenario-specific templates would constitute an extensive research endeavor that was explicitly beyond the intended scope of this thesis.

Thus, while this thesis has thoroughly addressed \textbf{RQ1}, clearly demonstrating both methodological rigor and practical feasibility of the derived generic \ac{dpp} architecture, these acknowledged limitations suggest meaningful avenues for further research, refinement, and practical enhancement of the developed conceptual framework.

\textbf{RQ2} directly addressed the regulatory uncertainties and technological selection complexities, aiming to precisely identify the functional and technological requirements of \ac{dpp} systems capable of effectively supporting \acrlong{ce} objectives. It further sought to identify and evaluate technologies most suitable for meeting these rigorously defined requirements. To achieve this, the research pursued two complementary methodological avenues: a structured literature review and expert-based empirical assessment.

Firstly, the foundational requirements of qualifying \ac{dpp} systems were thoroughly identified through an in-depth literature review. Central to this effort were key regulatory analyses, notably informed by the \ac{espr} regulation and the requirement analysis provided by \textcite{Jansen.2023}. These analyses systematically elucidated mandatory and recommended data elements, interoperability conditions, lifecycle coverage expectations, and many more factors, thereby establishing a rigorous baseline of \ac{dpp} system requirements directly aligned with regulatory mandates and circular economy principles.

Building upon this literature-derived foundation, a structured qualitative expert evaluation was subsequently conducted, employing semi-structured interviews. These interviews served dual purposes: firstly, validating and refining the set of regulatory and functional requirements derived from the literature, and secondly, enabling the expert-driven derivation and prioritization of \ac{kpi}s essential for a robust, multi-dimensional \acrlong{uva}. \ac{kpi}s and their corresponding sub-\ac{kpi}s were carefully formulated based on existing literature and then methodically presented to industry and academic experts to assign their relative importance. This method provided a comprehensive, empirically grounded prioritization framework, allowing for nuanced assessment and comparative evaluation of system architectures.

In parallel, the exploration of suitable technologies involved both a targeted literature review of core research papers for this thesis and an expert-driven qualitative exploration. Experts were invited to discuss openly their experiences and insights regarding promising technological solutions and approaches for \ac{dpp} systems. This hybrid analytical approach allowed not only the validation of previously identified enabling technologies such as blockchain, digital twin frameworks (specifically the \ac{aas}), semantic modeling, and others, but also introduced industry insights on current technological maturity levels, practical adoption barriers, and anticipated evolutionary trends. Such insights significantly enriched the technological evaluation framework and ensured practical relevance and alignment with industry readiness and capacity.

Despite the methodological rigor and comprehensive empirical grounding, several inherent limitations emerged from the approach utilized to address \textbf{RQ2}. Firstly, the identified regulatory and functional requirements predominantly reflect the perspectives of regulators, policy experts, and \ac{dpp} specialists. While these stakeholder groups provide valuable insights, the voices and experiences of manufacturers, supply chain actors, and especially small and medium enterprises (SMEs) remained relatively underrepresented. Although partially captured through reviewed literature, direct input from these stakeholders, who face practical operational challenges in implementing \ac{dpp} systems, would provide additional layers of granularity, realism, and operational specificity to the derived requirements. Given that industry actors are still in the early stages of understanding and integrating \ac{dpp} systems, their limited participation highlights an ongoing gap and represents a critical limitation of this research and indeed of current \ac{dpp} research more broadly.

Secondly, while the expert interviews played a crucial role in validating and weighting the proposed \ac{kpi}s, the process revealed valuable suggestions for improvement, including recommendations for adding missing \ac{kpi}s (and/or sub-\ac{kpi}s), merging redundant ones, or excluding less relevant ones. However, these valuable amendments and insights were not incorporated during this research due to methodological constraints. Altering the structured interview format mid-study would have compromised the consistency, comparability, and uniformity of results necessary for rigorous \ac{uva}. Conducting an additional validation round with a revised \ac{kpi} set, although potentially enriching the research outcomes, fell beyond the intended scope and available resources of this study. Nevertheless, this limitation clearly indicates a meaningful avenue for future research, allowing for deeper iterative refinement and validation of the developed \ac{dpp} evaluation framework.

Last but not least, although a robust evaluation framework was methodically developed and validated, time constraints prevented its extensive practical application. In particular, a comprehensive evaluation and comparative analysis of existing system architectures, such as those highlighted in the core technology papers in \Cref{sec:technology_review} and some of the initiatives analyzed in \Cref{sec:data_analysis}, fell outside the scope of this study. Applying the \ac{uva} framework to those architectures would have provided additional validation, enhanced the generalizability of the findings, and yielded deeper insights into the relative strengths, weaknesses, and practical viability of both theoretical and real-world \ac{dpp} solutions. 

Overall, while the comprehensive approach undertaken to address \textbf{RQ2} succeeded in systematically identifying and evaluating core regulatory, functional, and technological requirements and solutions, the aforementioned limitations and considerations provide explicit pathways and valuable guidance for future research aiming to further refine, deepen, and operationalize the \ac{dpp} evaluation framework.

\textbf{RQ3} focused explicitly on addressing the practical challenge of implementing an effective and scalable \ac{dpp} system. It aimed to evaluate how the identified and modeled technological solutions could be practically realized, while critically examining the primary factors influencing their industry adoption. To empirically validate this, the pilot system architecture, extensively documented in \Cref{cha:chapter_5}, is subjected to a structured \ac{uva} according to the previously defined evaluation framework.

The \ac{uva}, as documented in \Cref{sec:expert_insights_evaluation}, involves assigning explicit performance scores (\(P_i\)) to the implemented pilot system across each \ac{kpi}. These scores were carefully derived through an internal analytical assessment informed by the thorough literature review, technical implementation experience, and considerations from previous research insights. Each \ac{kpi} score was allocated on a scale of 1 to 10, where higher scores reflect superior performance concerning the defined criteria. The performance scores assigned, along with their justifications, are as follows:

\begin{itemize}[itemsep=0.5\baselineskip]
    \item \textbf{Interoperability}: – \textit{10}\\
    The implemented system leverages the \acrlong{aas}, inherently designed around interoperability principles. Strict adherence to standardized \ac{idta} submodel templates, robust RESTful \ac{api}s following industry-best practices, and dynamic mapping capabilities of heterogeneous data streams within the digital twin architecture fully justify the highest possible rating.

    \item \textbf{Scalability}: – \textit{9}\\
    Scalability was ensured through a fully containerized implementation using Docker Compose, combined with Traefik reverse proxy for dynamic load balancing and seamless horizontal scaling. The architecture is production-ready from an infrastructure perspective. However, achieving maximum industrial scalability necessitates additional decentralization or federation of governance and data infrastructures, leading to a slightly reduced rating.

    \item \textbf{Data Security and Integrity}: – \textit{5}\\
    Standard cryptographic mechanisms ensure basic security; however, advanced data integrity measures such as Decentralized Identity (\ac{did}), \acrlong{vc}, logging mechanisms, or blockchain-based immutability were explicitly excluded from the pilot's initial scope. These exclusions, while intentional to maintain system simplicity and focus on core \ac{dpp} functionalities, significantly constrain the security rating.

    \item \textbf{Portability}: – \textit{8}\\
    Containerization significantly enhances portability, facilitating rapid and consistent deployment across diverse environments. Nonetheless, the necessity for custom data mappings from existing infrastructures to the digital twin remains, presenting a slight barrier to complete portability and justifying the assigned score.

    \item \textbf{Modularity}: – \textit{9}\\
    Modular design was a fundamental architectural principle, realized through the independent containerization of frontend, backend, proxy, and database components and separation of internal logical modules. While highly modular, a fully decoupled microservice backend or micro-frontend architecture, which would represent the ultimate modularity standard, remains beyond the scope of the current proof-of-concept implementation.

    \item \textbf{Usability}: – \textit{7}\\
    The Public \ac{dpp} Viewer component, intuitively designed for end-user accessibility, scores exceptionally high. However, the administrative \ac{aas} management dashboard, though robust, currently lacks critical advanced functionalities necessary for seamless integration in industrial processes, such as a custom submodel template editor and a user-friendly digital twin data pipeline configuration interface.

    \item \textbf{Reliability}: – \textit{6}\\
    Conceptually, the architecture provides a solid foundation for high reliability, particularly if deployed on cloud-native, auto-scaling infrastructures. However, without explicit production deployment, reliability assessment remains theoretical. Moreover, reliance on manual employee-managed data entry without blockchain-based immutable ledgers to guarantee accuracy further limits reliability confidence at this stage.

    \item \textbf{Cost-Effectiveness}: – \textit{9}\\
    Initial deployment and configuration of the data pipeline, integrating data from existing infrastructures such as \ac{erp}, \ac{plm}, \ac{iot} and other data stores into the digital twin submodel structures, incur moderate setup costs, predominantly related to IT resources. Post-deployment, the architecture's automated and dynamic nature significantly reduces ongoing operational costs. Additionally, the current \ac{qr} Code implementation for physical products is low-cost.

    \item \textbf{Compliance}: – \textit{10}\\
    The pilot is firmly rooted in comprehensive \ac{espr} research, with its technical requirements directly derived from regulatory mandates. By incorporating dedicated submodel templates and \ac{dpp} sections, it aligns fully with circular economy objectives and covers every stage of the product lifecycle—including extensive end-of-life documentation such as usage data, \ac{iot} integration, and R-strategy insights.
\end{itemize}

Having established these explicit performance scores, the overall utility (\(U\)) of the pilot system architecture can now be rigorously computed using the weighted sum of performances (for the calculated weights, refer to \Cref{fig:expert_kpi_calculations}).

\begin{align*}
\gls{sym:utility} 
&= (0.12 \times 10) + (0.09 \times 9) + (0.13 \times 5) + (0.09 \times 8) \\
&\quad + (0.11 \times 9) + (0.13 \times 7) + (0.11 \times 6) + (0.09 \times 9) + (0.12 \times 10) \\
&= 1.2 + 0.81 + 0.65 + 0.72 + 0.99 + 0.91 + 0.66 + 0.81 + 1.2 \\
&= 7.95
\end{align*}

The calculated overall utility score of 7.95 (79.5\%) represents a significantly high performance outcome, especially considering the implemented system was developed explicitly as a proof-of-concept pilot. This indicates that the pilot effectively addresses most key requirements defined in the evaluation framework and validates the practical feasibility and industrial relevance of the system. Notably, the demonstrated capability of the pilot system to rapidly transition organizations from no digital twin infrastructure to a fully operational digital twin integrated with an automatically generated \ac{dpp} represents a significant strength.

Furthermore, the direct application of the evaluation framework through the \ac{uva} provides a robust empirical validation of both the developed pilot and the utility-based evaluation methodology itself.

While this empirical evaluation approach provided rigorous analytical insights, it inherently carries certain methodological limitations. Most notably, the performance scores were derived through internal analytical assessments informed primarily by literature review, practical implementation experience, and theoretical considerations. Although this ensures a cohesive and research-informed evaluation, the robustness and external validity of these scores could have been considerably enhanced by incorporating structured expert evaluations or, even more ideally, deploying the pilot system as a test implementation within a real-world industry scenario. Such practical validation would allow for capturing direct industry feedback, refining the performance scores through firsthand usage experiences, and providing deeper insights into practical adoption barriers, user acceptance, and real-world effectiveness and scalability. Due to scope and resource constraints, such an extensive empirical validation was beyond this thesis.

\section{Conclusion and Future Outlook}

This research addressed the pressing and complex challenges inherent in the practical implementation of \ac{dpp} systems, guided explicitly by the research questions defined at the outset. Through comprehensive examination of the current state-of-the-art, empirical analysis of industry initiatives, and detailed expert-driven evaluation, this thesis developed an expert-informed universal evaluation framework for \ac{dpp} systems, a robust morphological box coupled with a generic architectural model, and validated these findings through the practical implementation of a highly modular \ac{dpp} pilot system.

The structured evaluation framework derived in this research provides industry stakeholders with a robust and empirically validated method to systematically assess and select optimal technologies, aligning both with regulations and circular economy principles. The morphological box and generic model, grounded in methodical rigor and industry best practices, deliver an invaluable toolset enabling tailored, yet cohesive, architectural configurations adaptable across diverse sector-specific scenarios. Finally, the proof-of-concept implementation explicitly demonstrated the real-world viability and effectiveness of the theoretical concepts, simultaneously exposing critical strengths and limitations that will guide further refinement and adaptation.

Nonetheless, several critical avenues for future research and development clearly emerged from this work. Firstly, technological evaluation, although comprehensive, inherently represents a snapshot of the rapidly evolving technological landscape. Given the fast-paced innovation and maturation of core digital technologies such as blockchain, semantic web, and \ac{iot}, continual reassessment and updating of technological evaluations will remain crucial for sustaining the framework’s relevance and accuracy.

Secondly, the expert-derived \ac{kpi}s utilized within this study highlighted meaningful suggestions and refinements not fully incorporated due to methodological constraints. Incorporating these adjustments and conducting additional expert validation rounds present a valuable next step to further enhance the robustness and practical alignment of the evaluation framework.

Thirdly, systematically compiling and rigorously validating representative sets of \ac{dpp} system architectures derived from all feasible permutations within the morphological model would significantly enhance its practical utility. Providing clear architectural templates for common industry scenarios would further reduce implementation barriers, streamline design decisions, and encourage broader industry adoption.

On a technical level, future development efforts should focus on integrating advanced functionalities such as an \ac{aas} submodel template designer and intuitive data mapping tools to facilitate seamless connections between existing infrastructures and digital twin architectures. Additionally, evolving the current modular design toward a fully decoupled microservice and micro-frontend architecture would enhance system modularity, maintainability, and scalability.

Finally, deploying the developed pilot within an actual industry environment and conducting extensive empirical validation remains an essential next step toward broad adoption, practical refinement, and deeper understanding of real-world effectiveness, usability, and industry acceptance.

The pilot system is the culmination of all the comprehensive methodological approaches employed in this research. It powerfully demonstrates the potential to transform unstructured industrial data systems into fully operational digital twins with compliant \ac{dpp} solutions in just a few minutes.